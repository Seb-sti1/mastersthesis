\documentclass[aspectratio=169,hyperref={pdfpagelabels=false}]{beamer}
\usepackage{microtype}      % better looking text
\usepackage[utf8]{inputenc}
\usepackage{fontspec}       % Package for custom fonts
\usepackage[]{geometry}     % Package for changing page margins (before fancyhdr)
\usepackage{fancyhdr}       % Package to change header and footer
\usepackage{parskip}        % Package to tweak paragraph skipping (instead of indents, a small skip is added after every paragraph)
\usepackage{titlesec}
\usepackage{tikz}           % Package for drawing
\usepackage{pgfplots}       % Package for creating graphs and charts
\usepackage{xcolor}         % Package for defining DTU colours to be used
\usepackage{amsmath}        % For aligning equations among other
\usepackage{siunitx}        % SI units
\usepackage{listings}       % Package for inserting code (before cleveref)
\PassOptionsToPackage{hyphens}{url} % Ability to line break urls at hyphens
\usepackage{hyperref}       % Package for cross-referencing (also loads url package)
\usepackage{cleveref}       % improved cross-referencing
\usepackage{textcomp}       % \textdegree = °C and other useful symbols
\usepackage{blindtext}

% main language (can be danish for BSc and BsE)
\usepackage[english]{babel} % localisation

\usepackage{caption}        % better captions
\usepackage{subcaption}     % for subfigures
\usepackage{csquotes}       % For biblatex with babel
\usepackage[backend=biber,style=numeric,sorting=none]{biblatex} % Package for bibliography (citing)
\bibliography{appendices/mastersthesis}
\usepackage{tabularx}       % for the ability to adjust column spacing in tabular better
\usepackage{booktabs}       % for better tables
\usepackage{float}          % floating figures in correct places
\usepackage{calc}           % Adds ability for latex to calculate (3pt+2pt)
\usepackage{relsize}
% some electronic packages
\usetikzlibrary{shapes,arrows,arrows.meta, positioning, snakes}
% \usetikzlibrary{positioning,calc,snakes} % decorations
\usepackage[RPvoltages, europeanresistor, arrowmos, americancurrents, americanvoltages]{circuitikz}
\usepackage{tikz}
\tikzset{>=triangle 45}
\usetikzlibrary{datavisualization}
\usetikzlibrary{datavisualization.formats.functions}
\usepackage[acronym]{glossaries} % add glossary and acronym list
\makeglossaries

\newacronym{uav}{UAV}{Unmanned Aerial Vehicule}
\newacronym{ugv}{UGV}{Unmanned Ground Vehicle}
\newacronym{csf}{CSF}{cloth simulation filter}
\newacronym{emd}{EMD}{earthmover's distance}
\newacronym{ai}{AI}{artificial intelligence}
\newacronym{bev}{BEV}{bird's-eye view}
\newacronym{mse}{MSE}{mean squared error}
\newacronym{ssim}{SSIM}{structural similarity index measure}

%\newglossaryentry{unmanned-aerial-vehicle}
%{
%    name=Unmanned Aerial Vehicle,
%    description={An aircraft without a human pilot on board}
%}
%\glsadd{unmanned-aerial-vehicle}
%
%\newglossaryentry{unmanned-ground-vehicle}
%{
%    name=Unmanned Ground Vehicle,
%    description={A vehicle that operates on the ground without a human onboard}
%}
%\glsadd{unmanned-ground-vehicle}

\newglossaryentry{rgbd}{
    name=RGBD,
    description={A color image (red, green, blue) including a channel for the depth in the camera frame.}
}

\newglossaryentry{lidar}{
    name=LiDAR,
    description={Light Detection and Ranging}
}

\newglossaryentry{ui}{
    name=UI,
    description={User interace}
}




\title{Collaborative navigation in unstructured environments using an aerial drone and a terrestrial robot}
\subtitle{Sébastien Yvon Claude Kerbourc'h (s232496)}

\setdepartment{DTU ELECTRO}
\setcolor{dtublue}

%\setbeameroption{show notes}
\setbeameroption{show notes on second screen=right}

\begin{document}
    \inserttitlepage
    \note{
        \begin{itemize}
            \item present the work i've done \textbf{during 5 months} at the \textbf{U2IS Lab in France}
            \item The subject is \textbf{collaborative navigations in unstructured environments using an aerial drone
            and a terrestrial robot}
            \item the goal in the master's thesis is to \textbf{use uav to enhance ugv's perception}
            \item \textbf{uav collects} data, structured in a topological map. \textbf{ugv local nav from node to node}
            \item \textbf{ugv recognize} when it reached a node with the uav data
            \item \textbf{difficulty: difference in perspective}
            \item
            \item
            \item autonomous navigation in unstructured terrains environments remains \textbf{far from viable}
            \item this does not come as a surprise as \textbf{complex systems rely on many subsystem} that any single failure
            \textbf{can comprise the entire system}
        \end{itemize}
    }

    \begin{frame} % table of content
        \tableofcontents[
            sectionstyle=show,
            subsectionstyle=hide,
            pausesections
        ]
        \note<1>{
            \begin{itemize}
                \item goal is to reduce the difference in perspective
                \item using the ugv camera to create a bev image: like if it were taken from above
            \end{itemize}
        }
        \note<2>{
            \begin{itemize}
                \item find keypoints that are in common between uav and ugv
                \item to be able to decide whenever two images from uav/ugv are the same
            \end{itemize}
        }
        \note<3>{
            \begin{itemize}
                \item using all the previous steps, simulate how it would work in the real world
                \item this was done on two datasets
            \end{itemize}
        }
    \end{frame}
    %
    % ================================================
    % ================================================
    % ================================================
    %


%    \section{Topological map creation}
%    \begin{frame}[tocremainder]
%        \tableofcontents[currentsection]
%        \note<1>{
%            \begin{itemize}
%                \item two methods
%                \item first based on Fast Color/Texture Segmentation For Outdoor Robots, Rufus Blas et al.
%                \item second based on detect tree Bosch et al.
%            \end{itemize}
%        }
%    \end{frame}

%
% ================================================
% ================================================
% ================================================
%


    \section{Creating bird’s-eye view from unmanned ground vehicule’s image}
    \begin{frame}[tocremainder]
        \tableofcontents[currentsection]
        \note<1>{
            \begin{itemize}
                \item lets starts with the perspective change
                \item i tried two methods
                \item the first is using rgb d (depth info) images
                \item the second is the more traditional approach of using a homography
            \end{itemize}
        }
    \end{frame}

    \subsection{Pointcloud reprojection of an RGBD image}
    \begin{frame}{Pointcloud reprojection of an RGBD image}
        \begin{columns}%
            \begin{column}{.5\textwidth}
                \begin{figure}
                    \centering
                    \includegraphics[width=0.99\linewidth]{illustrations/defence/reprojection}
                \end{figure}
            \end{column}%
            \begin{column}{.5\textwidth}
                \pause
                \begin{align}
                    u &= \dfrac{f}{z} x & \dfrac{u}{s_x} &= (j - c_x) \label{eq:pcd_rgbd:u} \\
                    v &= \dfrac{f}{z} y & \dfrac{v}{s_y} &= (i - c_y) \label{eq:pcd_rgbd:v}
                \end{align}
                \pause
                \begin{align}
                    x &= (j - c_x) \dfrac{z}{f_x} \label{eq:pcd_rgbd:x} \\
                    y &= (i - c_y) \dfrac{z}{f_y} \label{eq:pcd_rgbd:y}
                \end{align}
            \end{column}%
        \end{columns}
        \note<1>{
            \begin{itemize}
                \item point in the camera frame
                \item image frame, its projection
                \item x,y ; u,v ; i,j
                \item intercept theorem
                \item change of units
            \end{itemize}
        }
        \note<2>{
            \begin{itemize}
                \item intercept theorem
                \item change of units
            \end{itemize}
        }
        \note<2>{
            \begin{itemize}
                \item combine them
                \item depends only the calibration parameters, the pixel, and the depth
            \end{itemize}
        }
    \end{frame}

    \begin{frame}
        \begin{figure}[ht!]
            \centering
            \begin{subfigure}[t]{0.32\textwidth}
                \centering
                \includegraphics[width=0.99\linewidth]{illustrations/bev/zed_rgbd_rgb}
                \caption{The RGB image.}
                \label{fig:pcd_rgbd:rgb}
            \end{subfigure}
            \hfill
            \begin{subfigure}[t]{0.32\textwidth}
                \centering
                \includegraphics[width=0.99\linewidth]{illustrations/bev/zed_rgbd_d}
                \caption{The depth image.}
                \label{fig:pcd_rgbd:depth}
            \end{subfigure}
            \hfill
            \begin{subfigure}[t]{0.32\textwidth}
                \centering
                \includegraphics[width=0.99\linewidth]{illustrations/bev/zed_rgbd_pcd}
                \caption{The resulting colorized point cloud.}
                \label{fig:pcd_rgbd:pcd}
            \end{subfigure}
        \end{figure}
        \note<1>{
            \begin{itemize}
                \item for each pixel (using the zed calibration)
                \item ignore pixels where z is not valid
            \end{itemize}
        }
    \end{frame}

    \begin{frame}
        \begin{figure}
            \begin{subfigure}[t]{0.45\textwidth}
                \includegraphics[width=0.99\textwidth]{illustrations/bev/zed_rgbd_bev_0.05}
                \caption{column base width of 0.05m.}
            \end{subfigure}
            \hfill
            \pause
            \begin{subfigure}[t]{0.45\textwidth}
                \includegraphics[width=0.99\textwidth]{illustrations/bev/zed_rgbd_merged_cloud_bev_0.05}
                \caption{accumulated point cloud, column base width of 0.05m.}
            \end{subfigure}
            \hfill
        \end{figure}
        \note<1>{
            \begin{itemize}
                \item in the point cloud aggregate by columns, each column is one pixel of the bev
                \item aggregation is important, here the average is used
                \item really noisy... and sparse
                \item slow: Numba to parallelize the computation of each column.
                \item reduce it to about 750 ms, corresponding to an 11.8x speedup.
            \end{itemize}
        }
        \note<2>{
            \begin{itemize}
                \item fix this with accumulation of the point cloud, using icp
                \item filtering steps to try and reduce the size of the pointcloud
                \item alignment errors, causing structures like walls and grass to blend improperly.
                \item 100 000 points to 450 000 points, reducing the processing rate from around 1Hz to 0.5Hz.
                \item
                \item given those poor results, i tried another more traditional method: using an homography to deform the image
            \end{itemize}
        }
    \end{frame}

    \subsection{Traditional homography}

    \begin{frame}{Traditional homography}
        \begin{itemize}
            \item find the real world coordinates of A, B, C and D
        \end{itemize}
        \begin{columns}%
            \begin{column}{.5\textwidth}
                \begin{figure}
                    \centering
                    \includegraphics[width=0.99\linewidth]{illustrations/bev/homography}
                \end{figure}
            \end{column}%
            \pause
            \begin{column}{.5\textwidth}
                \begin{align}
                    \label{eq:homography:complete}
                    \begin{bmatrix}
                        x_r \\
                        y_r \\
                        0   \\
                        1
                    \end{bmatrix} &= T_{c\rightarrow r} \times
                    \begin{bmatrix}
                    (j - c_x)
                        \dfrac{z_c}{f_x}\\
                        (i - c_y) \dfrac{z_c}{f_y} \\
                        z_c                        \\
                        1
                    \end{bmatrix}
                \end{align}
            \end{column}%
        \end{columns}
        \note<1>{
            \begin{itemize}
                \item locally flat ground hypothesis
                \item $z_r = 0$
            \end{itemize}
        }
        \note<2>{
            \begin{itemize}
                \item using the previous equations
            \end{itemize}
        }
    \end{frame}


    \begin{frame}
        \begin{figure}[ht!]
            \begin{subfigure}[t]{0.35\textwidth}
                \includegraphics[width=0.99\textwidth]{illustrations/bev/homography_zed}
                \caption{The raw image from the ZED 2i.}
                \label{fig:homography:zed}
            \end{subfigure}
            \hfill
            \begin{subfigure}[t]{0.58\textwidth}
                \includegraphics[width=0.99\textwidth]{illustrations/bev/homography_zed_bev}
                \caption{The resulting BEV.}
                \label{fig:homography:zed_bev}
            \end{subfigure}
        \end{figure}
        \note<1>{
            \begin{itemize}
                \item much cleaner results
            \end{itemize}
        }
    \end{frame}
%
% ================================================
% ================================================
% ================================================
%


    \section{Finding correspondences between ugv and uav}
    \begin{frame}[tocremainder]
        \tableofcontents[currentsection]
        \note<1>{
            \begin{itemize}
                \item building on the previous work it is time to try and find correspondences between uav/ugv images
                \item jean michel fortin dataset
                \item diff between ideal and jmf dataset: ugv in uav view
            \end{itemize}
        }
    \end{frame}

    \subsection{Initial testing}
    \begin{frame}{Initial testing}
        \begin{itemize}
            \item Compare Xfeat, \pause Omniglue, \pause ORB at finding correspondences
            \pause
            \item First \("\)bruteforce\("\) attempt failed
            \pause
            \item Second attempt in an ideal scenario
        \end{itemize}
        \pause
        \begin{figure}[ht!]
            \centering
            \begin{subfigure}[t]{0.49\textwidth}
                \centering
                \includegraphics[width=0.99\linewidth]{illustrations/find_correspondances/x_feat_found3_kp}
                \caption{XFeat's results.}
                \label{fig:find_corr:along_traj:xfeat}
            \end{subfigure}
            \hfill
            \begin{subfigure}[t]{0.49\textwidth}
                \centering
                \includegraphics[width=0.99\linewidth]{illustrations/find_correspondances/omniglue_found1_kp}
                \caption{OmniGlue's results.}
                \label{fig:find_corr:along_traj:omniglue}
            \end{subfigure}
        \end{figure}
        \note<1>{
            \begin{itemize}
                \item Xfeat, CNN architecture, aims for semi-dense feature matching, efficiency.
                \item It processes high-resolution grayscale images through six convolutional blocks that reduce
                spatial resolution while increasing channel depth.
                The network outputs dense descriptors, reliability, and keypoint heatmaps.
                Keypoint detection uses a novel parallel branch, and matching employs a lightweight refinement module.
                XFeat is trained in a supervised manner with pixel-level ground truth correspondences from the MegaDepth and COCO\_20k subset of the COCO2017 dataset.
                Additionally, a lighter version of LightGlue (lighterglue) was trained using features from XFeat.
                \item OmniGlue aims at generalization, for keypoints SuperPoint and DINOv2 for image descriptors.
                \item It first extracts fine-grained features with SuperPoint and broad features with DINOv2.
                It then builds intra- and inter-image keypoint graphs, using DINOv2 to guide inter-image associations.
                Information is propagated via attention, disentangling positional and appearance signals to improve generalization.
                Finally, refined descriptors are optimally matched to find correspondences between keypoints.
                OmniGlue is also trained in a supervised manner using ground truth from previous work.
            \end{itemize}
        }
        \note<2>{
            \begin{itemize}
                \item Xfeat, CNN architecture, aims for semi-dense feature matching, efficiency.
                \item It processes high-resolution grayscale images through six convolutional blocks that reduce
                spatial resolution while increasing channel depth.
                The network outputs dense descriptors, reliability, and keypoint heatmaps.
                Keypoint detection uses a novel parallel branch, and matching employs a lightweight refinement module.
                XFeat is trained in a supervised manner with pixel-level ground truth correspondences from the MegaDepth and COCO\_20k subset of the COCO2017 dataset.
                Additionally, a lighter version of LightGlue (lighterglue) was trained using features from XFeat.
                \item OmniGlue aims at generalization, for keypoints SuperPoint and DINOv2 for image descriptors.
                \item It first extracts fine-grained features with SuperPoint and broad features with DINOv2.
                It then builds intra- and inter-image keypoint graphs, using DINOv2 to guide inter-image associations.
                Information is propagated via attention, disentangling positional and appearance signals to improve generalization.
                Finally, refined descriptors are optimally matched to find correspondences between keypoints.
                OmniGlue is also trained in a supervised manner using ground truth from previous work.
            \end{itemize}
        }
        \note<3>{
            \begin{itemize}
                \item Xfeat, CNN architecture, aims for semi-dense feature matching, efficiency.
                \item It processes high-resolution grayscale images through six convolutional blocks that reduce
                spatial resolution while increasing channel depth.
                The network outputs dense descriptors, reliability, and keypoint heatmaps.
                Keypoint detection uses a novel parallel branch, and matching employs a lightweight refinement module.
                XFeat is trained in a supervised manner with pixel-level ground truth correspondences from the MegaDepth and COCO\_20k subset of the COCO2017 dataset.
                Additionally, a lighter version of LightGlue (lighterglue) was trained using features from XFeat.
                \item OmniGlue aims at generalization, for keypoints SuperPoint and DINOv2 for image descriptors.
                \item It first extracts fine-grained features with SuperPoint and broad features with DINOv2.
                It then builds intra- and inter-image keypoint graphs, using DINOv2 to guide inter-image associations.
                Information is propagated via attention, disentangling positional and appearance signals to improve generalization.
                Finally, refined descriptors are optimally matched to find correspondences between keypoints.
                OmniGlue is also trained in a supervised manner using ground truth from previous work.
            \end{itemize}
        }
        \note<4>{
            \begin{itemize}
                \item brute force attempt: generate many patches in uav/ugv view and try to show a clear correlation
                between distance and number of correspondences
                \item patches because of omniglue (too much VRAM for full images)
            \end{itemize}
        }
        \note<5>{
            \begin{itemize}
                \item second attempt: ensure as much as possible that the extracted patch are the same
                \item align patch angle with the aruco tag on top of the ugv and the ugv in the uav frame
                \item take the same place in the image
                \item take uav/ugv patches with the same width
            \end{itemize}
        }
        \note<5>{
            \begin{itemize}
                \item promising initial results
            \end{itemize}
        }
    \end{frame}

    \begin{frame}
        \begin{columns}
            \begin{column}{.4\textwidth}
                \begin{itemize}
                    \item \textit{obviously wrong},
                    \pause
                    \item mean squared error (MSE),
                    \pause
                    \item structural similarity index measure (SSIM).
                \end{itemize}
            \end{column}%
            \begin{column}{.6\textwidth}
                \begin{figure}
                    \centering
                    \includegraphics[width=0.99\textwidth]{illustrations/defence/ssim}
                \end{figure}
            \end{column}%
        \end{columns}
        \note<1>{
            \begin{itemize}
                \item common areas in uav/ugv images are less than 10000 pixels$^2$,
                any reprojected quadrilateral is crossed, stretching the quadrilaterals to a square give
                more than 20 pixels completely black
                \item obviously wrong to ensure next indicators works
            \end{itemize}
        }
    \end{frame}

    \subsection{Test robustness to relative rotations}
    \begin{frame}{Test robustness to relative rotations}
        \begin{figure}
            \centering
            \includegraphics[width=0.83\linewidth]{illustrations/defence/angles_obviously_wrong}
            \caption{Obviously wrong ratios.}
        \end{figure}
        \vspace*{-1em}% TODO border issue
        \begin{itemize}
            \item $0\%$ is the best, $100\%$ is the worst.
        \end{itemize}
    \end{frame}

    \begin{frame}
        \begin{figure}[ht!]
            \centering
            \begin{subfigure}[t]{0.32\textwidth}
                \includegraphics[width=0.99\linewidth]{illustrations/find_correspondances/big_angle_0_scores_xfeat+lighterglue}
                \caption{XFeat (ligherglue), no rotation.}
                \label{fig:find_corr:big_angles:mse_ssim_xfeat+lg_0}
            \end{subfigure}
            \hfill
            \begin{subfigure}[t]{0.32\textwidth}
                \includegraphics[width=0.99\linewidth]{illustrations/find_correspondances/big_angle_90_scores_xfeat+lighterglue}
                \caption{XFeat (ligherglue), 90°.}
                \label{fig:find_corr:big_angles:mse_ssim_xfeat+lg_90}
            \end{subfigure}
            \hfill
            \begin{subfigure}[t]{0.32\textwidth}
                \includegraphics[width=0.99\linewidth]{illustrations/find_correspondances/big_angle_0_scores_omniglue}
                \caption{OmniGlue, no rotation.}
                \label{fig:find_corr:big_angles:mse_ssim_omniglue_0}
            \end{subfigure}
        \end{figure}
        \begin{itemize}
            \item For MSE, the lower, the better. For SSIM, 1 is the best, 0 is the worst.
            \item Not robust to rotation in general
        \end{itemize}
    \end{frame}


    \begin{frame}
        \begin{figure}[ht!]
            \centering
            \begin{subfigure}[t]{0.32\textwidth}
                \includegraphics[width=0.99\linewidth]{illustrations/find_correspondances/small_angle_0_scores_xfeat+lighterglue}
                \caption{XFeat (ligherglue), rotation}
                \label{fig:find_corr:small_angles:mse_ssim_xfeat+lg_0}
            \end{subfigure}
            \hfill
            \begin{subfigure}[t]{0.32\textwidth}
                \includegraphics[width=0.99\linewidth]{illustrations/find_correspondances/small_angle_5_scores_xfeat+lighterglue}
                \caption{XFeat (ligherglue), 5°.}
                \label{fig:find_corr:small_angles:mse_ssim_xfeat+lg_5}
            \end{subfigure}
            \hfill
            \begin{subfigure}[t]{0.32\textwidth}
                \includegraphics[width=0.99\linewidth]{illustrations/find_correspondances/small_angle_-10_scores_xfeat+lighterglue}
                \caption{XFeat (ligherglue), -10°.}
                \label{fig:find_corr:small_angles:mse_ssim_xfeat+lg_-10}
            \end{subfigure}

            \begin{itemize}
                \item For MSE, the lower, the better. For SSIM, 1 is the best, 0 is the worst.
                \item Robust to small rotation -10° - 10°.
            \end{itemize}
        \end{figure}
        \note<1>{
            \begin{itemize}
                \item 0 not the same: method to compute the angle not the same
            \end{itemize}
        }
    \end{frame}
%
% ================================================
% ================================================
% ================================================
%


    \section{Simulation of real world scenario}
    \begin{frame}[tocremainder]
        \tableofcontents[currentsection]
        \note<1>{
            \begin{itemize}
                \item now that we have a way to find correspondences between the two images uav/ugv
            \end{itemize}
        }
    \end{frame}

    \subsection{Implementation}
    \begin{frame}{Implementation}
        \begin{itemize}
            \item Graph is a collection of interconnected nodes
        \end{itemize}
        \begin{figure}
            \centering
            \includegraphics[width=0.35\linewidth]{illustrations/system_test/montmorency_graph}
            \caption{Topological map for the Montmorency.}
        \end{figure}
        \note<1>{
            \begin{itemize}
                \item node: name, gnss coordinate of center, radius, array of patches, features of best patches
            \end{itemize}
        }
    \end{frame}

    \begin{frame}{Implementation}
        \begin{itemize}
            \item \textbf{Generate scouting data}: 4x7 grid of patches, oriented at 0°
            \item Each patches in a node is extracted
            \pause
            \item \textbf{Filter scouting data}: keep the 10 best patches
            \item Use median of score of features, minimal distance between patches
            \pause
            \item \textbf{Detect UGV location}: knowing the next node, generate correspondences
            \item Two patches with more than 600 correspondences $\Rightarrow$ same location
        \end{itemize}
        \note<1>{
            \begin{itemize}
                \item oriented at 0° = east
                \item GNSS coordinates center = pixel positions in the image, the UAV’s position and orientation, image scale
            \end{itemize}
        }
        \note<2>{
            \begin{itemize}
                \item median of scores because there seemed to be a small correlation
                \item minimal distance to reduce duplication
            \end{itemize}
        }
    \end{frame}

    \subsection{Results using Montmorency dataset}
    \begin{frame}{Results using Montmorency dataset}
        \begin{adjustwidth}{-3.5em}{-1.5em}
            \begin{table}[ht!]
                \centering
                \begin{tabular}{|>{\raggedright\arraybackslash}p{2.5cm}|c|c|c|c|c|c|c|c|c|c|c|c|c|}
                    \hline
                    Node                & 1  & 2  & 3 & 4  & 5    & 6  & 7    & 8 & 9 & 10   & 11 & 12 & 13   \\ \hline
                    Detection Rate (\%) & 80 & 60 & 0 & 30 & 33.3 & 25 & 88.9 & 0 & 0 & 44.4 & 10 & 80 & 22.2 \\ \hline
                \end{tabular}
                \caption{Patch detection rate for each node.}
            \end{table}%
            \begin{table}[ht]
                \centering
                \begin{tabular}{|>{\raggedright\arraybackslash}p{4cm}|c|c|c|c|c|c|c|c|c|c|c|c|c|}
                    \hline
                    Node                                      & 1 & 2 & 3 & 4 & 5 & 6 & 7 & 8 & 9 & 10 & 11 & 12 & 13 \\ \hline
                    Number of time robot goes through node    & 3 & 2 & 2 & 2 & 2 & 1 & 2 & 1 & 3 & 1  & 1  & 1  & 1  \\ \hline
                    Number of time robot correctly detects it & 3 & 2 & 0 & 1 & 2 & 1 & 2 & 0 & 0 & 1  & 1  & 1  & 1  \\ \hline
                \end{tabular}
                \caption{Node traversal and detection counts.}
            \end{table}%
        \end{adjustwidth}
        \note<1>{
            \begin{itemize}
                \item UGV itself appears within the image frames
                \item exclusion zone
                \item sometimes captures marks of the UGV's wheels
                \item only GNSS position and the orientation of the UGV
                \item UAV position is recomputed (ArUco comes handy)
            \end{itemize}
        }
    \end{frame}

    \subsection{Results using a new U2IS dataset}
    \begin{frame}{Results using a new U2IS dataset}
        \begin{adjustwidth}{-3.5em}{-1.5em}
            \begin{table}[ht!]
                \centering
                \begin{tabular}{|l|c|c|c|c|}
                    \hline
                    Node                & 1  & 2  & 3 & 4 \\ \hline
                    Detection Rate (\%) & 50 & 50 & 0 & 0 \\ \hline
                \end{tabular}
                \caption{Patch detection rate for each node.}
            \end{table}%
            \begin{table}[ht]
                \centering
                \begin{tabular}{|l|c|c|c|c|}
                    \hline
                    Node                                      & 1 & 2 & 3 & 4 \\ \hline
                    Number of time robot goes through node    & 1 & 1 & 1 & 1 \\ \hline
                    Number of time robot correctly detects it & 1 & 1 & 0 & 0 \\ \hline
                \end{tabular}
                \caption{Node traversal and detection counts.}
            \end{table}%
        \end{adjustwidth}
        \note<1>{
            \begin{itemize}
                \item one patch was created for both the UAV and the UGV
                \item scaling of the images
                \item 2.8 times larger in the UGV's BEV than in the UAV's image
                \item best solution was found to be scaling up the UAV
            \end{itemize}
        }
    \end{frame}

    \begin{frame}
        \begin{adjustwidth}{-3.5em}{-1.5em}
            \begin{table}[ht!]
                \centering
                \begin{tabular}{|l|c|c|c|c|c|}
                    \hline
                    Node                & 0   & 1  & 2  & 3 & 4    \\ \hline
                    Detection Rate (\%) & 100 & 25 & 50 & 0 & 12.5 \\ \hline
                \end{tabular}
                \caption{Patch detection rate for each node.}
            \end{table}%
            \begin{table}[ht]
                \centering
                \begin{tabular}{|l|c|c|c|c|c|}
                    \hline
                    Node                                      & 0 & 1 & 2 & 3 & 4 \\ \hline
                    Number of time robot goes through node    & 1 & 1 & 1 & 1 & 1 \\ \hline
                    Number of time robot correctly detects it & 1 & 1 & 1 & 0 & 1 \\ \hline
                \end{tabular}
                \caption{Node traversal and detection counts.}
            \end{table}%
        \end{adjustwidth}
        \note<1>{
            \begin{itemize}
                \item what i changed
            \end{itemize}
        }
    \end{frame}

    \subsection{Limitations and future works}
    \begin{frame}{Limitations and future works}
        \begin{itemize}
            \item Backbone of the topological map
            \pause
            \item Using homography to create BEV has limitations
            \pause
            \item Xfeat (lighterglue) needs scale to be close
            \pause
            \item Better indicators
            \pause
            \item Better filtering of patches
        \end{itemize}
        \note<1>{
            \begin{itemize}
                \item Still requires to create the topological map
            \end{itemize}
        }
        \note<2>{
            \begin{itemize}
                \item BEV no flat ground. but results really good
            \end{itemize}
        }
        \note<3>{
            \begin{itemize}
                \item Xfeat (lighterglue) needs scale to be close
            \end{itemize}
        }
        \note<4>{
            \begin{itemize}
                \item better indicators
                \item ideal: true positives, true negatives, false positives, and false negatives
                \item but true negatives hard
                \item higher level indicators?
            \end{itemize}
        }
        \note<5>{
            \begin{itemize}
                \item filtering of the patches to select a better coverage of the patches area
                \item threshold okay but use more of the available data better
            \end{itemize}
        }
    \end{frame}


    \section{}
%
% ================================================
% ================================================
% ================================================
%
    \setbeamercolor{background canvas}{bg = dtublue}
    \begin{frame}[plain, noframenumbering]{}
        \usebeamerfont{title}
        \usebeamercolor[fg]{title}

% title
        \begin{tikzpicture}[remember picture,overlay]
            \node[anchor=south west,
                text width = \paperwidth,
                xshift=0mm,
                yshift=10mm]
            at (current page.south west)
                {{\usebeamerfont{subtitle}%
            \usebeamercolor[fg]{subtitle}%
            Thank you for your attention,} \\
            {\textbf{\color{white}Any questions?}}};
        \end{tikzpicture}
    \end{frame}
    \setbeamercolor{background canvas}{bg = white}
%
% ================================================
% ================================================
% ================================================
%
    \begin{frame}{Fast Color/Texture Segmentation For Outdoor Robots}
        \begin{itemize}
            \item k-means clusterization using descriptors based on CIE-Lab $\rightarrow$ \textit{textons},
            \begin{align}
                p_{i,j} &= \begin{bmatrix}
                               W_1 \times L_c         \\
                               W_2 \times a_c         \\
                               W_2 \times b_c         \\
                               W_3 \times (L_1 - L_c) \\
                               \vdots                 \\
                               W_3 \times (L_8 - L_c) \\
                \end{bmatrix}
            \end{align}
            \item \textit{textons} representative of local color and texture features,
            \item create histograms of the \textit{textons} in a neighborhood window.
        \end{itemize}
        \note<1>{
            \begin{itemize}
                \item k means will find k clusters of point closest in the descriptors space
                \item $W_{i}$ are weights to change the importance of color/texture
                \item $L$ the lightness, $a$ and $b$ for color
                \item
                \item fine-grained (micro-level) description of the image,
                \item ensures that the algorithm can adapt to different environments
                \item generating a representation tailored to the specific visual characteristics of the input.
                \item
                \item use the local descriptors to cluster in term of composition of descriptors
            \end{itemize}
        }
    \end{frame}

    \begin{frame}{Fast Color/Texture Segmentation For Outdoor Robots}
        \begin{figure}[ht!]
            \centering
            \begin{subfigure}[t]{.29\textwidth}
                \includegraphics[width=.99\textwidth]{illustrations/topological_map_creation/fast_color_raw1}
                \caption{Raw image used as input}
            \end{subfigure}
            \hfill
            \begin{subfigure}[t]{.69\textwidth}
                \includegraphics[width=0.99\textwidth]{illustrations/topological_map_creation/fast_color_hist1}
                \caption{2D histogram for three given \textit{textons}. The greener, the higher the density of the \textit{texton}.}
                \label{fig:fast_color:histograms}
            \end{subfigure}
        \end{figure}
        \begin{itemize}
            \item start to reveal structural information,
            \item new descriptor vector using the histograms.
        \end{itemize}
        \note<1>{
            \begin{itemize}
                \item start to reveal structural information:
                \item the first histogram shows high density over grassy areas and low density over paths
                \item the second appears to emphasize trees
                \item the third again highlights grassy regions.
                \item
                \item create new descriptor vector where each coordinate are the histograms of the k textons
                \item k means again
            \end{itemize}
        }
    \end{frame}

    \begin{frame}
        \begin{itemize}
            \item EMD to merge clusters too close
        \end{itemize}
        \begin{figure}
            \begin{subfigure}[t]{0.45\textwidth}
                \includegraphics[width=0.99\textwidth]{illustrations/topological_map_creation/fast_color_before_emd}
                \caption{The clusters before merging using the EMD. minimal distance of 0.}
                \label{fig:fast_color:before_emd}
            \end{subfigure}
            \hfill
            \begin{subfigure}[t]{0.45\textwidth}
                \includegraphics[width=0.99\textwidth]{illustrations/topological_map_creation/fast_color_emd_10}
                \caption{Final clusters using EMD minimal distance of 10.}
                \label{fig:fast_color:emd_10}
            \end{subfigure}
            \hfill
        \end{figure}
        \note<1>{
            \begin{itemize}
                \item emd to merge clusters (in the histograms descriptor space) too close (earth mover's distance)
                \item there are clusters that could be linked to specific areas like grass/path
                \item but overall it is not sufficient to generate a topological map
            \end{itemize}
        }
    \end{frame}

    \begin{frame}{Detecting trees in order to find traversable region}
        \begin{itemize}
            \item supervised learning to train an Adaboost classifier with an 27 features long vector
        \end{itemize}
        \begin{figure}
            \begin{subfigure}[t]{.32\textwidth}
                \includegraphics[width=0.99\linewidth]{illustrations/topological_map_creation/detect_tree_b50_r10000}
                \caption{$\beta = 50$, $\text{rescale} = 10000$ (default)}
            \end{subfigure}
            \hfill
            \begin{subfigure}[t]{.32\textwidth}
                \includegraphics[width=0.99\linewidth]{illustrations/topological_map_creation/detect_tree_b1000_r10000}
                \caption{$\beta = 1000$, $\text{rescale} = 10000$}
            \end{subfigure}
            \hfill
            \begin{subfigure}[t]{.32\textwidth}
                \includegraphics[width=0.99\linewidth]{illustrations/topological_map_creation/detect_tree_b1000_r30000}
                \caption{$\beta = 1000$, $\text{rescale} = 30000$}
            \end{subfigure}
        \end{figure}
        \note<1>{
            \begin{itemize}
                \item other technics U-Net... no dataset...
                \item hypothesis that a non tree region is traversable (it contains a traversable path)
                \item GIST descriptors, k means clusters the tiles to reduce the quantity of data to label (k is chosen so that the number of clusters is about 1\% the number of tiles)
                \item supervised learning to train an Adaboost classifier with an 27 features long vector
                \item 27 features = 6 encoding color, 18 texture, 3 entropy
                \item limited it seems more important to work on perspective change and determining if two images are the same place
            \end{itemize}
        }
    \end{frame}
\end{document}