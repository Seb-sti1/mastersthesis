\chapter{Conclusion}\label{ch:conclusion}

%\textcolor{dtured}{ The conclusion should list the positive results of the thesis and set the results in the perspective of the objectives.}
%
%\textcolor{dtured}{(Future work - things not finished or not working - should preferably be somewhere else, \textit{e.g.} in the previous chapter).}

This thesis addressed the challenges of enabling an \gls{ugv} to localize and navigate using data gathered by an \gls{uav} in unstructured environments.
The proposed system centers around constructing a topological map from \gls{uav} imagery by identifying key features such as paths and intersections.
Techniques were introduced to mitigate the perspective differences between aerial and ground views, including the generation
of \glspl{bev} from \gls{ugv} images and appropriate scale normalization.


The construction of the topological map backbone, primarily using segmentation techniques to identify paths and intersections,
yielded early results.
Although it did not fully meet expectations in terms of automation and robustness, the current progress lays a
foundation for future development.
Further work is needed to refine these techniques and achieve reliable, fully automated topological map generation.

Two algorithms were explored to reduce the perspective differences between \gls{uav} and \gls{ugv} data.
While the first approach produced poor results, the second unexpectedly provided satisfactory outcomes.
Naturally, limitations arise when operating outside the assumptions of the model, but the results suggest significant
potential for real-world applicability.

The use of sparse correspondence methods proved particularly well-suited to this work.
Despite requiring specific conditions related to viewing angle and scale, the generalization capability of the approach
was more than satisfactory and demonstrated a robust foundation for matching features across modalities.

Finally, the application of the topological map for scouting, filtering, and localization tasks functioned as intended.
The simulated real-world scenario yielded very encouraging results, which can likely be improved further with relatively minor adjustments.

Overall, these early-stage developments appear promising and suggest that the proposed methods could realistically be
implemented on real robotic systems operating in real-world environments.
