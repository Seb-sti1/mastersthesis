\chapter{Introduction}\label{ch:introduction}

The idea for this thesis originated from the observation that autonomous navigation in outdoor environments
remains far from viable.
This is not surprising, as true autonomous navigation in complex, unstructured environments
requires multiple robust systems and algorithms—any of which, if failing, can compromise the entire system.

For the autonomous navigation to work, the robot requires notably a precise estimation of its localization,
a way to perceive the environment, a way to establish a plan using those raw data and something that can
execute the plan.

At each step, there are challenges.
For the localization, no one sensor can provide accurate and high frequency data.
This lead the aggregation of multiple sources of localization that come with different noise.

Perceiving the environment, especially negative objects (holes in the ground), can also be quite challenging.
A range of sensors (e.g.\ LiDAR, RGBD Camera etc.) are used to sense the world around the robot.
These raw datas are then combined and process to produce higher level of understand of the environment.

With this understanding of the environment, new steps allows to create a plan for the robot to execute.
To ensure reactivity to new information received and/or a changing environment, this planning along with the
already mentioned systems, need to be constantly refreshed.

Finally, with all these processes the robot can execute the elaborated plan.


\section{Background}\label{sec:background}

traversability

local nav

localization

start to be okay ish


\section{The project}\label{sec:the-project}

Local navigation is a necessary first step toward autonomous navigation.
However, it must be complemented by global navigation, which provides overall direction and guides the robot toward its final goal.
This global navigation won't have the same level of the detail, it will happen at a higher level of abstraction.
Its objective is not, contrary to the local navigation, to provide velocity command to perform for the robot, but
to give a rough path towards the end goal.


some use satelite vikingnav


This idea behind this project is to use an \gls{uav} to gather terrain data like roads, crossroads, obstacles, distinctive
elements\ldots This features would then be used, depending on the objective point to reach, to determine a global
path and recognizable places for the \gls{ugv} to locate.


\section{State of the art}\label{sec:state-of-the-art}

use projectplan, extend on other articles


\section{Context}\label{sec:context}

This master's thesis took place at the U2IS, the computer science lab of ENSTA Paris.
The lab has some robots and sensors with each of them having their particularities.

Also, while I was working at the lab, my team was participating in a robotics challenge named CoHoMa.
This challenge aims at making a fleet of UAV and UGV move and recognize red boxes on an unstructured terrain.


robot available in u2is lab

tundra drone with gimbal and jetson

barakuda robot with lidar, ptz, zed, jetson

zed is not inclined a lot

difficulty of making dataset

\subsection{Robots}

The main UGV of the lab are now two \href{https://www.shark-robotics.com/robots/barakuda-mule-robot}{Barakuda from SharkRobotics}
and the main UAV are two \href{https://www.hexadrone.fr/produits/drone-tundra/}{Tundra 1 from hexadrone}.

% TODO add pictures of barakuda

On the Barakuda multiple sensors are added.
Two cameras, the \href{https://www.stereolabs.com/en-fr/products/zed-2}{ZED 2i} and the \href{https://www.axis.com/products/axis-q62-series}{AXIS Q62 PTZ}
are added to the robot.
The ZED 2i is attached at the front of the robot in order to obtain \Gls{rgbd} image of the robot immediate surroundings.
On the other hand, the PTZ, which is a controllable camera (along pitch and yaw axes), is placed about XX cm above the center of the
robot.
% TODO fill the heigh of the PTZ
Currently, the main purpose of this camera is to give a better perspective to the user when the robot is used in
teleoperation mode.
Additionally, to these two cameras, a \Gls{lidar} is added at the front a robot.
On one of the Barakuda it is an \href{https://ouster.com/products/hardware/osdome-lidar-sensor}{OSDome} while on the
other one it is an \href{https://ouster.com/products/hardware/os1-lidar-sensor}{OS1}.
All these sensors are connected to an NVIDIA Jetson Orin AGX\@.

% TODO add schematics of barakuda
% TODO add jetson spec in appendix?

On the Jetson is running docker containers running ROS noetic.
The docker containers are needed as the software environment on the Jetson, especially for libraries, can be difficult
to get right.
Inside the container, ROS noetic nodes are running to enable the robot to sense, process, and react to command and the
environment.




detailed explanation of the robots: sensors, software, how then interact

\subsection{CoHoMa}

current challenges/direction of the lab


\section{Objectives}