\chapter{Introduction}\label{ch:introduction}
\glsresetall

The idea for this thesis originated from the observation that autonomous navigation in outdoor environments remains far from viable, especially in real-world, unstructured terrains
This is not surprising, as such autonomy demands a combination of reliable systems and algorithms, all operating simultaneously and consistently---any single failure can compromise the entire system.

For autonomous navigation to be functional, several core capabilities are required.
The robot must be able to localize itself accurately, perceive its environment effectively, generate a plan based on this information, and physically execute that plan.
Each of these components introduces its own challenges and complexity.

In localization, for instance, no single sensor provides both the accuracy and high frequency needed across all environments.
\gls{gnss} is usually too infrequent, while \glspl{imu} and odometry suffer from drift.
This necessitates the fusion of multiple sensor inputs---each with its own noise profile and limitations---into a consistent and reliable estimate of the robot’s pose, often through complex filtering or optimization methods.

Environmental perception is equally demanding, particularly for detecting negative obstacles such as holes, ditches,
or sudden drops, which are inherently harder to sense.
A combination of sensors such as \gls{lidar}, stereo or \gls{rgbd} cameras is typically used to gather spatial information.
However, raw data alone is insufficient---processing pipelines must filter, fuse, and interpret these data streams into
higher-level representations like elevation maps or obstacle grids, which are then usable by the planning algorithms.

With an understanding of the terrain and obstacles, the robot must be able to generate a feasible plan that accounts for
both short-term reactivity and long-term goals.
This requires layered planning architectures, typically involving a global planner for strategic pathfinding and a local
planner for real-time adjustments.
Because the environment can change unpredictably or new information may become available, both perception and planning
modules must operate continuously and update their outputs at high frequency.

Finally, with all these processes in place, the robot must execute the elaborated plan through its actuators,
which serve as the interface between computation and physical action.
Actuators must accurately translate velocity and steering commands into real-world movement, despite variability in
terrain, load, and mechanical resistance.
In practice, real-time feedback from low-level sensors like encoders is used to monitor and adjust actuation.
In challenging outdoor conditions, reliable execution is critical, as poor control can quickly degrade the performance
of even the most sophisticated planning system.

% {\color{dtured}Introduction, start at the level of any high-school student and end in broad terms with the purpose and content of this thesis.}


\section{The project}\label{sec:the-project}

This project focuses on global navigation, treating local navigation as fixed and outside its scope.
Global navigation operates at a higher level of abstraction than local navigation and does not require the same level of detail
understanding of the robot surrounding.
Its role is not to generate low-level velocity commands for the robot but to provide an overall path or sequence of
waypoints that guide the robot toward its final goal.
It acts as a strategic coordinator for local navigation, which is then responsible for real-time adjustments and obstacle avoidance.
Naturally, the more traversable the global path is, the easier it becomes for local navigation to execute it successfully.

The objective of this thesis is to explore techniques to construct the backbone of a topological map from aerial reconnaissance.
This includes identifying paths and intersections from \gls{uav} imagery and defining how they can be represented within a navigable map structure.
Subsequently, the focus shifts to enabling cooperation between the \gls{uav} and \gls{ugv}.
Using different algorithms, each robot collects and transforms the sensor data available to it.
The goal is to allow the reconnaissance performed by the \gls{uav} to support the \gls{ugv} in identifying its location
within the topological map and navigating through it.


\section{Objectives}

\begin{itemize}
    \item Develop an algorithm to identify paths and intersections in unstructured terrain.
    \item Design techniques to reduce perspective differences between the \gls{uav} and the \gls{ugv}.
    \item Develop algorithms to identify common areas in both \gls{uav} and \gls{ugv} data.
    \item Implement a computationally efficient correspondence-finding system to recognize areas detected by the \gls{uav} from the \gls{ugv} perspective.
    \item Ensure and quantify the reliability of the proposed implementation.
\end{itemize}


\section{Limitations}\label{sec:limitations}

Several important limitations affect this project.
First, all datasets and experiments are constrained to the robots described in \Cref{ch:context},
meaning no additional sensors or hardware modifications can be used to address the objectives.
Furthermore, local navigation is outside the scope of this thesis; existing software components, also detailed in \Cref{ch:context},
are used as-is, with their respective advantages and shortcomings.
Due to regulatory constraints, \gls{uav} flights must be performed under strict conditions---limited to specific locations,
times, and flight height---which significantly restricts the possibility of creating \gls{uav}-collected datasets.
Lastly, during the thesis period, the \gls{u2is} participated in the CoHoMa III robotics challenge, occasionally making some robots unavailable for use.


\section{Thesis outline}

The following chapters explore several key aspects of this thesis:

\begin{itemize}
    \item \textbf{Analysis} --- Reviews the current state of the art and introduces the specific challenges addressed in this work.
    \item \textbf{The Context of the Project} --- Describes the equipment available at the \gls{u2is}, including hardware configurations and existing software components.
    \item \textbf{Topological Map Creation} --- Presents techniques for segmenting \gls{uav} imagery to identify paths and intersections, forming the basis for constructing a topological map.
    \item \textbf{Creating \gls{bev} from \gls{ugv} Images} --- Evaluates two methods for generating \gls{bev} images from \gls{ugv}-mounted cameras.
    \item \textbf{Finding Correspondences between \gls{ugv} and \gls{uav} Data} --- Explores several algorithms for establishing sparse correspondences between \gls{uav} and \gls{ugv} data.
    \item \textbf{System Test} --- Implements and evaluates a simulated real world
    scenario using two different datasets and compare it against prior results.
\end{itemize}

