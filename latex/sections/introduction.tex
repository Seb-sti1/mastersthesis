\chapter{Introduction}\label{ch:introduction}

The idea for this thesis originated from the observation that autonomous navigation in outdoor environments
remains far from viable.
This is not surprising, as true autonomous navigation in complex, unstructured environments
requires multiple robust systems and algorithms—any of which, if failing, can compromise the entire system.

For the autonomous navigation to work, the robot requires notably a precise estimation of its localization,
a way to perceive the environment, a way to establish a plan using those raw data and something that can
execute the plan.

At each step, there are challenges.
For the localization, no one sensor can provide accurate and high frequency data.
This lead the aggregation of multiple sources of localization that come with different noise.

Perceiving the environment, especially negative objects (holes in the ground), can also be quite challenging.
A range of sensors (e.g.\ LiDAR, RGBD Camera etc.) are used to sense the world around the robot.
These raw datas are then combined and process to produce higher level of understand of the environment.

With this understanding of the environment, new steps allows to create a plan for the robot to execute.
To ensure reactivity to new information received and/or a changing environment, this planning along with the
already mentioned systems, need to be constantly refreshed.

Finally, with all these processes the robot can execute the elaborated plan.


\section{Background}\label{sec:background}

traversability

local nav

localization

start to be okay ish


\section{The project}\label{sec:the-project}

Local navigation is a necessary first step toward autonomous navigation.
However, it must be complemented by global navigation, which provides overall direction and guides the robot toward its final goal.
This global navigation won't have the same level of the detail, it will happen at a higher level of abstraction.
Its objective is not, contrary to the local navigation, to provide velocity command to perform for the robot, but
to give a rough path towards the end goal.


This idea behind this project is to use an \gls{uav} to gather terrain data like roads, crossroads, obstacles, distinctive
elements\ldots This features would then be used, depending on the objective point to reach, to determine a global
path and recognizable places for the \gls{ugv} to locate.


\section{Objectives}

define


\section{Limitations}\label{sec:limitations}


robot available in u2is lab

difficulty of making dataset

cohoma

\section{Thesis outline}

