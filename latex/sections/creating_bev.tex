\chapter{Creating \gls{bev} from \gls{ugv}'s image}\label{ch:creating-bev-from-ugv's-image}


% TODO check the following paragraph
As discussed in~\cref{ch:analysis} and more specifically in~\cref{sec:problem-at-hand} one of the main challenge of
finding correspondences between \gls{ugv} and \gls{uav} data is the difference in perspective.
Indeed, the \gls{ugv} sees from the ground and is inevitably close to it while the \gls{uav} has more a god's eye
perspective and has a more global picture of its surrounding.
This very difficulty is also what makes so interesting to create a system that can leverage both point of view.

In the literature, the usual way to reduce the gap in perspective in the imagery between the \gls{ugv} and the \gls{uav} is to transform
the images in what is called a \gls{bev}.
As its name implies, the goal is to transform the image so that it appears like it was taken from above, like what a
bird eye would see.\\
There are several methods to this, the following subsections explore two tries of creating a \gls{bev} from the
\gls{ugv}'s image.


\section{Pointcloud reprojection of an RGBD image}

On possibility to obtain a \gls{bev} image is to use a \gls{rgbd} image, project its associated point cloud and then
flatten the point cloud from the top perspective.
Here, this is possible as the ground robots have a ZED 2i embedded as the front.
The intrinsics parameters of the ZED $f_x$, $f_y$, $c_x$, $c_y$, respectively the focal along x/y-axes of the camera
and the optical center (all in pixels), are given by the camera as additional metadata.
Using a pinhole camera model, the associated coordinates in the camera frame of each pixel are computed as follows.
Let the origin of the coordinate system be the centre of projection, then the image plane, where the image forms, is at $z = f$.
Using the intercept theorem the point $\begin{bmatrix} % TODO add triangle intercept?
                                           x & y & z
\end{bmatrix}^T$ is projected at $\begin{bmatrix}
                                      \dfrac{f}{z} x & \dfrac{f}{z} y & f
\end{bmatrix}^T$.
Let the 2D coordinates on the image plane be $u$, $v$ along, respectively, the x and y axes and $i$, $j$ be the pixel
coordinates.
Finally, let $s_x$ and $s_y$ be, respectively the horizontal and vertical dimension of one pixel of the sensor array.
Using these notations and the previous results give \cref{eq:pcd_rgbd:u} and \cref{eq:pcd_rgbd:v}.%
\begin{align}
    u &= \dfrac{f}{z} x & \dfrac{u}{s_x} &= (j - c_x) \label{eq:pcd_rgbd:u} \\
    v &= \dfrac{f}{z} y & \dfrac{v}{s_y} &= (i - c_y) \label{eq:pcd_rgbd:v}
\end{align}
Combining each set of equations and that by definition $f_x = \dfrac{f}{s_x}$ and $f_y = \dfrac{f}{s_y}$ gives the final result.%
\begin{align}
    x &= (j - c_x) \dfrac{z}{f_x} \label{eq:pcd_rgbd:x} \\
    y &= (i - c_y) \dfrac{z}{f_y} \label{eq:pcd_rgbd:y}
\end{align}
As underlined by \cref{eq:pcd_rgbd:x} and \cref{eq:pcd_rgbd:y}, only the position of the pixel in the image, the intrinsics
parameters and the depth are required to compute the actual coordinate in the camera frame of the said pixel.
This is why using a \gls{rgbd} camera is important as it gives the depth of each pixel.
These equations result in, for instance, create the colorized point cloud shown in \cref{fig:pcd_rgbd:pcd} using
\cref{fig:pcd_rgbd:rgb} and \cref{fig:pcd_rgbd:depth} as inputs.

\begin{figure}[ht!]
    \centering
    \begin{subfigure}[t]{0.32\textwidth}
        \centering
        \includegraphics[width=0.99\linewidth]{illustrations/bev/zed_rgbd_rgb}
        \caption{The RGB image.}
        \label{fig:pcd_rgbd:rgb}
    \end{subfigure}
    \hfill
    \begin{subfigure}[t]{0.32\textwidth}
        \centering
        \includegraphics[width=0.99\linewidth]{illustrations/bev/zed_rgbd_d}
        \caption{The depth image.}
        \label{fig:pcd_rgbd:depth}
    \end{subfigure}
    \hfill
    \begin{subfigure}[t]{0.32\textwidth}
        \centering
        \includegraphics[width=0.99\linewidth]{illustrations/bev/zed_rgbd_pcd}
        \caption{The resulting colorized point cloud.}
        \label{fig:pcd_rgbd:pcd}
    \end{subfigure}
    \caption{Example of reconstruction of the point cloud using an RGBD image.}
    \label{fig:pcd_rgbd:pcd_construction}
\end{figure}

Once the point cloud is computed, it needs to be flattened to a 2D image from the top perspective.
A simple way to do that is to consider rectangle columns with a square base.
The points, representing pixels, in the columns will be aggregated into one pixel.
The smaller the base is the more detailed the resulting \gls{bev} is.

One parameter that can greatly influence the result is the aggregation function used.
The main goal here is to ensure that resulting \gls{bev} image keeps as much of the recognisable details.
The choice that was made is to use an average of all the pixels in the column.
On the one hand, one of the advantage of using an average, unlike for instance using the highest pixel, is that a single outlier will not affect the end results.
On the other hand, it will give new colors, not existing in the original image, when a column has different object present in it which is not ideal.
Other aggregation functions, like median or histograms to select the most representative pixel, were considered but the first (and only)
attempt was using an average.

The first major inconvenient arise from the use of python.
Python is very good to quickly create small snippet of code but is far from perfect when trying to obtain an efficient algorithm.
The current implementation, using numpy and other libraries, is taking slightly less than 9s on average for each input image with a base width of ten centimeters.
Using~\cite{lam_numba_2015} to parallelize the computation of each column reduced it down to about 750 ms, corresponding to an 11.8x speedup.

\begin{figure}[ht!]
    \centering
    \includegraphics[width=0.5\textwidth]{illustrations/bev/zed_rgbd_bev_0.05}
    \caption{The \gls{bev} obtained with a base width of 0.05m}
    \label{fig:pcd_rgbd:bev_0.05}
\end{figure}

Following all the described steps with \cref{fig:pcd_rgbd:rgb} and \cref{fig:pcd_rgbd:depth} results in \cref{fig:pcd_rgbd:bev_0.05}.
Particular objects present in the image can be distinguished in the bev : the white walls, the shadow of the tree and the tree itself.
The \gls{bev} is very (very) noisy, which is not satisfactory at all.
For instance, noise in the point cloud close to the tree branches create a white trace starting around the branches and ending at the right wall.
Another unacceptable issue is the sparseness of the image as the density of pixels drastically fall after about six meters and a half.

So, to try and fix these issue, a new version of the technique was implemented.
This time instead of only using points from one image, it accumulated them across multiple images similar to one of the weekly project
done in \textit{Perception for Autonomous Systems}. % 34759 Perception for autonomous systems Weekly Project Day 9 2024
To ensure proper alignment between two consecutive point cloud, ICP~\parencite{besl_method_1992} is used to line them up.
Then two filtering steps are added to reduce the final size of the point cloud.
The first one exclude points too far away from the center and the second apply a voxel sampling with a size of $\dfrac{\text{column base width}}{2}$.

\begin{figure}[ht!]
    \centering
    \includegraphics[width=0.4\textwidth]{illustrations/bev/zed_rgbd_merged_cloud_bev_0.05} % TODO add the accumulated bev
    \caption{A \gls{bev} generated using an accumulated point cloud and a column base width of 0.05m}
    \label{fig:pcd_rgbd:accumulated_bev_0.05}
\end{figure}

Sadly, after couple hundred iterations corresponding to about twenty secondes, the point cloud rise from hundred
thousand points to four hundred fifty points, reducing the computation speed from about 1hz to 0.5Hz, and the ICP does not
manage to keep the point cloud enough aligned.
This results in poor \gls{bev}, with one example shown in \cref{fig:pcd_rgbd:accumulated_bev_0.05}.
It shows that the issue of sparseness is resolved by the accumulation, but it creates massive shift aberration as the walls and
the grass are mixing.

Therefore, as these techniques did not yield the expected results, a new simpler technic based on using an homography to
transform the image was implemented.


\section{Traditional homography}

% TODO finish code for this

do the code and finish this