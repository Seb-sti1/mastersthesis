\chapter{Analysis}\label{ch:analysis}

The subject can be decomposed into several layers, each with its own complexity.
The first task is for the drone to perform reconnaissance of the area of interest.
During this phase, the \gls{uav} must identify features to construct a topological map.
These features must be selected such that the \gls{ugv} can later use them for localization.

Once the \gls{uav} mapping is complete, the \gls{ugv} receives the topological map and begins navigation.
During navigation, the \gls{ugv} uses its sensors—potentially after preprocessing—and compares its measurements with the data collected by the \gls{uav}.


\section{State of the art}\label{sec:state-of-the-art}

% TODO preamble
% {\color{dtured}This chapter analyses the thesis objectives. Takes the overall decisions that set the frame for the following chapters. The following chapters can therefore concentrate on the chapter subject, as the (primary) relation to other chapters is handled in this analysis chapter.
% This chapter is to describe, assesses, and decides the overall alternatives for the approach to this project.
% It can describe, in some detail, the background for the project, the environment where it fits, and other items relevant to making the overall approach decisions for the project.

\subsection{Terrain segmentation or classification}

Many papers worked on segmenting or even classifying terrain types.
Some uses \gls{uav} imagery~\parencite{khan_visual_2012} while other uses satellite imagery~\parencite{sofman_terrain_2006},
the main difference between the two approaches being the resolution (pixels/ground meters) of the classification.
As discussed in~\ref{sec:problem-at-hand} and mainly because of the current lab activities~\ref{sec:current-lab-activities},
the most commonly encountered terrain are off-road, forest-like, unstructured environment.
In these types of terrain, forest trail like path are usually present, which is what ~\cite{rasmussen_appearance_2009} (using ~\cite{rufus_blas_fast_2008})
and~\cite{giusti_machine_2016} are trying to detect, in order to help robot's local navigation.
Another way to detect places where the robot can navigate through using \gls{bev} images is to detect forest~\parencite{bosch_journal_2020}
and consider that, on a first approach, non-forest area can be traversed using the local detection of the remaining obstacles.\\
As will be discussed later one, these approach, even if able give interesting data, usually are not able to give
sufficient results to find paths and intersections from the \gls{uav}'s point of view.

\subsection{Collaboration}\label{subsec:collaboration}

Regarding the collaboration part between \gls{uav} and \gls{ugv}, a lot of recent papers are exploring what is possible.
~\cite{delmerico_active_2017} shows an UAV scouting terrain information for the \gls{ugv} to reach a particular goal.
Its main limitation comes from the constraint of having to do a 3D reconstruction of the area to take the topography
into account.
On the contrary, their on the spot classifier~\parencite{kulic_--spot_2017} classifier is interesting as it allows
a lot of flexibility for the variety of terrain that can be handled.\\
Similarly,~\cite{fortin_uav-assisted_2024} shows that it is possible to create a machine learning model that predicts
vibration, bumpiness and power consumption of the \gls{ugv} using \gls{bev} from the \gls{uav}.
The trained model can then be used by the UAV to scout terrain traversability beforehand.
In~\cite{zhang_dual-bev_2025}, the \gls{ugv}'s camera is used to create a \gls{bev} of the surrounding before extracting traversability features.
It, then, uses overhead maps to select the optimum path to reach the objective.
These examples clearly demonstrate how well a \gls{bev} can be leverage to improve local navigation.\\
Even when the objective is to use the sensor datas to locate the robot at a node, using \gls{bev} allows to reduce
the difference in perspective between the \gls{uav} and the \gls{ugv}.
This difference is the main challenge that sets this work apart from solution similar to the one proposed in~\cite{han_effective_2020}.

% TODO some use satelite vikingnav

% TODO 3D reconstruction issue (cite \cite{wang_2d_2021} ?)

% TODO \cite{munasinghe_comprehensive_2024} everywhere

% TODO read \cite{wang_vggt_2025}
% TODO read \cite{deng_spatiotemporal_2025}
% TODO read \cite{sun_road_2025}


\section{Problem at hand}\label{sec:problem-at-hand}

%\section{Decisions}
%{\color{dtured}
%Explain overall decisions and which - more detailed - requirements this set for the work packages needed to meet the project objectives.
%}
%\section{Summary}
%{\color{dtured}
%Summarize the work needed to meet the project objectives, preferably in blocks corresponding to the later chapters.
%}

segmentation -> lot of different cases

uav constraints and advantages
- not a lot of computing power
- high (constraints and advantage): helpful to keep com and far from the ground
- better perspective to understand the environment
- fewer sensors (weight constraints)

ugv constraints and advantages
- more lot of computing power
- on the ground (constraints and advantage): problem keep com and close to the ground
- more sensors


terrain types
%    In my case, the goal is to improve the global navigation while keeping the amount of computation power and storage
%    in a reasonable range.

com issues

why topological map

hypothesis on the ugv/uav
- able to locally locate between to node
- able to navigate from one node to the other (under the assumption that nodes are generated in a way that render it possible)

