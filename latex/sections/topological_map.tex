\chapter{Topological map creation}\label{ch:topological-map-creation}

As discussed in~\ref{sec:problem-at-hand}, the main purpose of using a topological map is to retain only the necessary information about the environment.
A traditional map scales quadratically with the area’s width, whereas a topological map grows linearly with the number of nodes.
These nodes, typically representing intersections, are the key locations where information is stored.
The number of generated node is an important subject as it is the main parameter to regulate the final map size.
This approach relies on the assumption that the \gls{ugv} can use local navigation to reach a target when the required trajectory is approximately a straight line.


\section{Fast Color/Texture Segmentation For Outdoor Robots}

In~\cite{rasmussen_appearance_2009}, the team at the university of Delaware, uses the method describe in~\cite{rufus_blas_fast_2008},
lidar data and geometric constraints, in order to detect a forest trail and make an \gls{ugv} follow it.\\
As both articles seemed to gave promising results, a first attempt to segment \gls{uav} imagery using the method describe in
~\cite{rufus_blas_fast_2008}.

As, at this point, as no dataset using U2IS' \gls{uav} was available, the publicly available~\cite{noauthor_aukerman_nodate} was used.
It is a dataset of the Aukerman Park\footnote{15561 York Rd, North Royalton, OH 44133, United States}, containing 78
4896 x 3672 images captured with a Sony DSC-WX220.
It was chosen as it contains trails, grass, trees, gravel/sand paths and asphalt roads from an aerial perspective which
seemed like a good variety of terrains (see \cref{ch:aukerman-park-overview}).

\cite{rufus_blas_fast_2008} article's method, starts by converting the input RGB image to the \textit{CIE-Lab} space.
Then for each pixel, descriptors are computed using~\cref{eq:fast_color:descriptor} where the first three features
describe color while the remaining eight describe the local texture.
\begin{align}
    \label{eq:fast_color:descriptor}
    p_{i,j} = \begin{bmatrix}
                  W_1 \times L_c         \\
                  W_2 \times a_c         \\
                  W_2 \times b_c         \\
                  W_3 \times (L_1 - L_c) \\
                  \vdots                 \\
                  W_3 \times (L_8 - L_c) \\
    \end{bmatrix}
\end{align}
The pixel descriptors are then clusterized using the well-known k-means algorithm, which returns clusters of similar
pixel (in the sense of close in the descriptor space) and their centroid, called \textit{textons}.
At this time a filtering of the outliers is applied to remove descriptor too far away from their \textit{textons}.
For each pixel, a new feature vector is created by looking at the repartition/histogram of the clusters in a local
window centered on the pixel.
In a sense the \textit{textons} describe the image at a micro level while the histograms describe the composition in terms
of quantity of each \textit{textons} in a local area.
Furthermore, the histograms are clustered again use k-means and a final step uses the earthmover's distance to
merge histogram cluster having a centroid too close to each other.

\begin{figure}
    \centering
    \includegraphics[width=5cm]{illustrations/topological_map_creation/fast_color_raw1}
\end{figure}

\begin{figure}
    \centering
    \includegraphics[width=0.7\textwidth]{illustrations/topological_map_creation/fast_color_hist1}
\end{figure}

% TODO show results
%      interesting frequency stuff
%      not good enough results


\section{Detecting trees in order to find traversable region}

absence of consequent properly formated dataset at the lab, segment anything retrained or~\cite{ronneberger_u-net_2015} trained (like done in~\cite{zhang_dual-bev_2025}
and other) is not viable

but there is~\cite{bosch_journal_2020}, pretrained to segment trees on satellite imagery.
A clever tricks is to segment trees and consider that everywhere there is no trees the robot would be able to traverse
or at least find a traversable path avoiding not exclude obstacle (e.g.\ negative obstacles, rocks, etc.).


% TODO show results
%      different resolution
%      post-processing
%      on satellite map (not good enough)

- good results but not enough to try to use that to create topological map backbone

- not that that much added value as it would easy doable by a user with a properly designed gui
