%! Author = seb-sti1
%! Date = 03/02/2025

\documentclass[11pt, a4paper]{article}

\usepackage[margin=25mm]{geometry}
\usepackage[utf8]{inputenc}
\usepackage{mlmodern}
\usepackage[T1]{fontenc}
\usepackage{mathtools}                  % Nice and easy math
\usepackage{amsfonts}                   % Insert math symbols
%\usepackage{pgffor}                    % Latex for-loops
%\usepackage[ruled,vlined]{algorithm2e}  % Generating algorithm boxes
\usepackage{parskip}                    % Good practice linebreaks
\usepackage{algorithm}
\usepackage{algorithmic}
\usepackage{amsmath, amssymb, fancyhdr, lastpage}
\usepackage{cancel, tikz, graphicx, subcaption, float, enumerate}
\usepackage[short]{optidef}
\usepackage{multirow}
\usepackage{arraycols}
\usepackage[bookmarks]{hyperref}
\usepackage{cleveref}

\usepackage[acronym]{glossaries} % add glossary and acronym list
\makeglossaries

\newacronym{bev}{BEV}{Bird's-eye view}



\usepackage{listings} % code snippets
\usepackage{placeins}
\usepackage[dvipsnames]{xcolor} % color for code snippets
\input{configs/listlisting}

% bibliography
\usepackage[backend=biber, style=authoryear, citestyle=authoryear]{biblatex}
\addbibresource{appendices/mastersthesis.bib}

\hypersetup{%
    colorlinks = true, % Colors the links, instead of creating a colored box around them
    linkcolor = {blue!30!black},
    citecolor = {blue!50!black},
    urlcolor = {blue!80!black}
}

\setcounter{tocdepth}{1}
\renewcommand{\baselinestretch}{1.2}
\parindent=0pt
\setlength{\headheight}{15pt}

\newcommand{\thesistitle}{Collaborative navigation in unstructured environments using an aerial drone and a terrestrial robot}
\title{\Huge \thesistitle}
\newcommand{\thesisauthor}{Sébastien Yvon Claude KERBOURC'H}
\author{\thesisauthor}
\newcommand{\mastersname}{Master of Science (MSc) in Autonomous Systems}
\date{February 2025 - July 2025}

\hypersetup{
    pdftitle={\thesistitle},
    pdfauthor={\thesisauthor},
    pdfsubject={\mastersname},
    pdfkeywords={Master's thesis},
    pdfcreator={\thesisauthor},
}


\newcommand{\frontpagetextcolour}{black} % front page text colour: white or black
\definecolor{frontbackcolor}{HTML}{FFFFFF}

\fancypagestyle{main}{
    \fancyhead{}
    \fancyfoot{}
    \renewcommand{\headrulewidth}{1px}
    \fancyhead[LE]{\footnotesize \leftmark}
    \fancyhead[LO]{\footnotesize \rightmark}
    \fancyhead[RE]{}
    \fancyhead[RO]{}
    \fancyfoot[LE,RO]{\footnotesize \thepage}
    \fancyfoot[RE,LO]{\footnotesize \thesistitle}
    \fancyhfoffset[E,O]{0pt}
}
\pagestyle{main}

\fancypagestyle{plain}{
    \fancyhead{}
    \fancyfoot{}
    \renewcommand{\headrulewidth}{0pt}
    \fancyfoot[LE,RO]{\footnotesize \thepage}
    \fancyfoot[RE,LO]{\footnotesize \thesistitle} % - \leftmark
    \fancyhfoffset[E,O]{0pt}
}

\makeatletter
\def\thickhrulefill{\leavevmode \leaders \hrule height 1ex \hfill \kern \z@}
\def\@makechapterhead#1{{%
    \vspace*{4em}%
    {\sffamily \Huge \textsc{\@chapapp}\space \thechapter \par\nobreak}%
    \vspace*{3em}%
    {\hfill \Huge \textsc{#1} \par\nobreak}%
    \vspace*{10\p@}%
    \hrule%
    \vskip 100\p@}}
\def\@makeschapterhead#1{} % Remove chapter for TOC, Glossary
\makeatother


%minted doesn't compile
%\usepackage{minted}
%\usepackage{gensymb}
%\setminted{breaklines, linenos}
%\usemintedstyle{manni}

\renewcommand{\subtitle}{\Large{Project Plan}}
\chead{Master's thesis - Project Plan}
\rhead{}

% Document
\begin{document}
    \begin{titlepage}
    \makeatletter
    \vspace*{-2cm}

    \noindent
    \begin{minipage}[t]{0.5\textwidth}
        \includegraphics[height=3cm]{logo/dtu}
    \end{minipage}%
    \hfill
    \begin{minipage}[t]{0.5\textwidth}
        \raggedleft
        \includegraphics[height=3cm]{logo/enstaparis}
    \end{minipage}

    \vspace*{5cm}
    \begin{center}
        \textsc{\@title}\\
        \vspace*{0.5cm}
        \textsc{\subtitle}\\
        \vspace*{0.5cm}
        \textsc{\location}\\
        \rule{\linewidth}{0.5mm}
        \vspace*{0.2cm}\\
        \large{\@author}
    \end{center}
    \vspace*{1.5cm}
    \begin{center}
        Supervisor : Søren HANSEN\\
        Co-supervisors : Alexandre CHAPOUTOT, Thibault TORALBA\\
        \vspace*{1cm}
        \@date
%        \vspace*{1.5cm}
%        Rapport non-confidentiel
    \end{center}
    \makeatother
\end{titlepage}


    \section{Initial research}\label{sec:initial-research}

    The first month of my master's thesis was focus toward looking at different papers, finding datasets and testing
    different snippets of code related to the subject.

    I first started looking at ways to segment maps either for unmanned aerial vehicle (UAV) images (e.g.~\cite{khan_visual_2012}) or satellite imagery
    (e.g.~\cite{sofman_terrain_2006}) in order to construct a topology map (of the traversable path) of the relevant area.
    I also found other techniques when looking at forest trail detection and segmentation (e.g.~\cite{giusti_machine_2016}
    or~\cite{rufus_blas_fast_2008} used in~\cite{rasmussen_appearance_2009}).
    Finally, I tested~\cite{bosch_journal_2020} that gave good results on UAV imagery and decent results on satellite imagery.
    This could, alone or along with other methods, help to create decent topology maps either using satellite imagery or UAV imagery.
    This topology map would contain information about the GNSS coordinates of each node and the distance of each edge.
    Other techniques (e.g. \cite{wang_2d_2021}) based on full 3D reconstruction exist, but,
    as the aim of this project is to run them onboard in real-time, they appear suboptimal.

    Focusing on the collaborative aspect,~\cite{delmerico_active_2017} shows an UAV scouting terrain information for
    the unmanned ground vehicle (UGV) to reach a particular goal.
    Similarly,~\cite{fortin_uav-assisted_2024} shows that it is possible to create a machine learning model that predicts
    vibration, bumpiness and power consumption of the UGV using Bird-eye view (BEV) from the UAV.
    The trained model can then be used by the UAV to scout terrain traversability beforehand.
    In~\cite{zhang_dual-bev_2025}, the UGV's camera is used to create a BEV of the surrounding before extracting
    traversability features.
    It, then, uses overhead maps to select the optimum path to reach the objective.
    These examples clearly demonstrate how well a BEV can be leverage to improve local navigation.

    In my case, the goal is to improve the global navigation while keeping the amount of computation power and storage
    in a reasonable range.
    Going back to the topology map (that will either be generated semi-automatically or manually),
    one of the objectives of this map would be to store information about each node,
    gathered by the UAV, in order for the UGV, once at that node, to be able to locate itself.
    In a sense, it would be similar to~\cite{han_effective_2020} if the references were images from a UAV's BEV
    and the images to match were images (potentially preprocessed or transformed) from the UGV.
    The complexity here will come from the differences in point of view (POV).


    \section{Main objectives going forward}\label{sec:main-objectives-going-forward}

    Given this early research, I will, going forward, explore various techniques to reproject the UGV's POV in a BEV.
    The goal here is to ensure that the transformation will not remove or reduce the perceptibility of visual
    features that could be matched in the UAV's BEV.
    Then, the UGV's BEV and UAV's BEV will need to be matched to determined if there are representing the same
    location.
    Techniques like~\cite{comport_accurate_2007} based on~\cite{comport_statistically_2006}, that can either be used
    with dense or sparse correspondences, could be a good place to start.
    More abstract techniques, where features are extracted using neural networks (e.g.~\cite{zhang_dual-bev_2025}),
    can also yield some results, given the creation of a representative dataset.

    \textit{See Gantt chart on the next page.}

    \newpage


    \begin{figure}[!ht]
        \centering
        \includegraphics[angle=90,height=24cm]{illustrations/gantt}
        \caption{Gantt}
        \label{fig:gantt}
    \end{figure}
    \FloatBarrier

    \newpage


    \section{References}\label{sec:references}

    \printbibliography[heading=none]
\end{document}