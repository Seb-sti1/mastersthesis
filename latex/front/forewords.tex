%======================== Copyright
\thispagestyle{empty}
\setcounter{page}{1}
\vspace*{\fill}

\textbf{\thesistitle} \newline
\thesissubtitle

\smallskip

%\documenttype \newline
%\thedate

\smallskip

By \newline
(\studentnumber) \thesisauthor

\bigskip

\textbf{Advisor(s)}\newline
\ThSupervisors

\vspace{3cm}

\begin{tabularx}{\textwidth}{@{}lX@{}}
    Education & \documenttype\\
    Programme & \programme\\[4mm]
    Period & \projectperiod\\[4mm]
    ECTS & \thesissize \\[4mm]
    Copyright: & \thesisauthor \, \thesisyear \\[2mm]
    & Reproduction of this publication in whole or in part must include the customary bibliographic citation, including author attribution, report title, etc. \\[9mm]
    Cover photo:  & \thesisauthor, 2025 \\[4mm] % yellow leafs
    Published by: & DTU, \departmentdescriber,                                                                                                                               \\
    & \addressI, \addressII ~Denmark                                                                                                                           \\
    & \url{\departmentwebsite} \\[4mm]
    Remark        & This report is submitted as partial fulfillment of the requirements for graduation in the above education at the Technical University of Denmark.\\[9mm]

% for phd thesis only
%    ISSN: & [0000-0000] (electronic version) \\
%    ISBN: & [000-00-0000-000-0] (electronic version) \\
%    & \\
%    ISSN: & [0000-0000] (printed version) \\
%    ISBN: & [000-00-0000-000-0] (printed version)
\end{tabularx}


%======================== Abstract
\clearpage
\pagestyle{main}
\section*{Abstract}\label{sec:abstract}
\addcontentsline{toc}{section}{Abstract}

%{\textcolor{dtured}{An abstract is a bit from the introduction, a bit about the work done and the main conclusion.}}

This project explores the collaboration between an aerial drone and a terrestrial robot for navigating unstructured environments.
The drone performs initial environmental mapping using sensors (e.g.\ GNSS, cameras\ldots), generating a topological map with identified paths, intersections, and potential targets.
The terrestrial robot receives high-level instructions (e.g., “reach a target”) and navigate using this topological map.
It utilizes onboard sensors to match environmental features previously detected with the drone and associated to the topological map.


%======================== Acknowledgements
\clearpage
\section*{Acknowledgements}\label{sec:acknowledgements}
\addcontentsline{toc}{section}{Acknowledgements}

I would like to express my gratitude to my supervisor, Søren Hansen, for allowing me to conduct my master’s thesis abroad in France.
The topic I was offered aligned perfectly with my interests, and it was a great opportunity to work on it.

I extend special thanks to Alexandre Chapoutot and Thibault Toralba, who thought of me for this subject.
I am grateful for their guidance and the trust they placed in me and my intuitions.

I also wish to thank Clément, Adrien, David, Rémi, Joris, and Danil,
with whom I had the pleasure of working, sharing problems, and finding solutions.

During my work, I used the publicly available~\cite{noauthor_aukerman_nodate} and the one used in
~\cite{fortin_uav-assisted_2024} very kindly provided by Jean-Michel Fortin of the Northern Robotics Laboratory.
I am very grateful that I could use them and thank the authors.

Finally, the overwhelming majority my thesis' work is available on \href{https://github.com/Seb-sti1/mastersthesis}{my github}.